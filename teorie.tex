\section*{Teoretická část}
Změříme ohniskovou vzdálenost tenké čočky Besselovou metodou a metodou dvojího zvětšení.
Při Besselově metodě umístíme předmět a stínítko do vzájemné vzdálenosti $D$. Posouváním spojné čočky určíme její dvě polohy, při kterých je obraz zaostřen na stínítko. Pokud je vzdálenost těchto dvou poloh čočky $\Delta$, pak ohnisková vzdálenost čočky $f$ je \cite{skripta}
\begin{equation}\label{e:bessel}
f=\frac{D^2-\Delta^2}{4D} \,.
\end{equation}

Při metodě dvojího zvětšení změříme příčné zvětšení $\beta$ při dvou různých vzdálenostech předmětu $a$ nebo obrazu $\apr$. Označíme-li veličiny odpovídající obou uspořádáním dolními indexy $1$ resp. $2$, pak ohnisková vzdálenost čočky je \cite{skripta}
\begin{equation}\label{e:dvojizvetseni}
f=\frac{\left|\apr_1-\apr_2\right|}{\left|\beta_2-\beta_1\right|}=\frac{\beta_1 \beta_2 \left|a_1-a_2\right|}{\left|\beta_2-\beta_1\right|} \,.
\end{equation}


Pokud spojnou čočkou nebudou procházet pouze paraxiální paprsky, zjistíme, že se protnou v jiném místě (viz \cite{skripta}). Vzdálenost průsečíku neparaxiálních paprsků od průsečíku paraxiálních paprsků nazýváme sférickou vadou čočky a značíme $v$, přičemž používáme stejnou znaménkovou konvenci. $v$ je funkce polohy předmětu $a$ a vzdálenosti paprsku od optické osy $s$. Pro jednotlivou čočku platí přibližně \cite{skripta}
\begin{equation} \label{e:sf_chyba}
v=Ks^2 \,,
\end{equation}
kde $K$ je konstanta. Pro spojnou čočku je $K<0$.


U čočky splývají hlavní body s uzlovými body, takže můžeme pomocí goniometru určit vzdálenost hlavních rovin $\delta$.
Ze známě tloušťky čočky $d$ potom můžeme určit index lomu skla $n$ pomocí vztahu \cite{skripta}
\begin{equation} \label{e:tlusta}
\delta = \frac{n-1}{n} d \,.
\end{equation}