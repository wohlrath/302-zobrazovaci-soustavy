\section*{Diskuze}
Při měření kulové vady jsme považovali clonu č. 1 za paraxiální, tedy kulovou vadu pro ostatní clony jsme měřili jako rozdíl mezi polohami obrazu první a oné clony. Vzniklá chyba je zanedbatelná v porovnání s ostatními vlivy.

Podle zadání jsme považovali za $s$ aritmetický průměr vnějšího a vnitřního poloměru clony. K tomuto postupu není zřejmý důvod, jako lepší se jeví místo aritmetického průměru použít kvadratický průměr nebo ještě lépe $s$ středované přes intenzitu světla. Obraz bude v každém případě rozostřen a my neznáme přesný mechanismus vyhodnocování jeho polohy a nemůžeme tedy rozhodnout, která z možností je nejlepší. Proto předpokládáme, že použití aritmetického průměru má dobrý důvod a je správné.

Kulová vada je u obou předmětových vzdáleností větší, když je čočka orientovaná ploskou stranou k předmětu, což kvalitativně odpovídá našim výpočtům. Kulová vada je větší pro menší z obou předmětových vzdáleností.

Při Besselově metodě jsme se dopustili systematické chyby, protože jsme zanedbali tloušťku čočky. $D$ ve skutečnosti není vzdálenost vzdálenost předmětu a obrazu, ale $D=a+\apr$, kde $a$ a $\apr$ se měří od hlavních rovin. Abychom dostali správnou hodnotu $D$, musíme od změřené odečíst vzdálenost hlavních rovin $\delta_{tenka}$. Potom by vyšla ohnisková vzdálenost \SI{10.69(10)}{\cm}, tedy jsme se dopustili systematické chyby přibližně \SI{1.3}{\percent}.

Při metodě dvojího zvětšení je vnesená systematická chyba zanedbatelná. Výpočet ze změny obrazových vzdáleností byl přesnější. V každé dvojici má vždy buď součin $\beta_1\beta_2$ nebo rozdíl $\left|a_1-a_b\right|$ velkou chybu.

Rozmezí obou hodnot optické moutnosti, které jsme změřili na fokometru, přibližně odpovídá změřené optické vzdálenosti.

Metoda dvojího zvětšení byla přesnější.

Index lomu skla vypočtený z tloušťky tlusté čočky a vzdálenosti jejích hlavních rovin je v rozsahu hodnot, kterých běžně nabývá. Tabelovaná hodnota je 1,5-1,9.